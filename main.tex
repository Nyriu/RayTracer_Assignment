\documentclass{article}
%documentclass[draft]{article}

%\usepackage[italian]{babel}
\usepackage[utf8]{inputenc}


\usepackage{graphicx} % Immagini fantastiche e...
\graphicspath{        % dove trovarle
  {./images/},
  {./images/2D/},
}

%% Bibliografia
\usepackage{csquotes}
\usepackage{biblatex}
\addbibresource{bibl.bib}


% Grafici direttamente in latex
\usepackage{tikz}
\usetikzlibrary{shapes,positioning,calc}
\colorlet{lightgray}{gray!20}

\usepackage{rotating} % per tabella ruotata
\usepackage{makecell}
%\usepackage[showframe=true]{geometry}
\usepackage{changepage}

% Immagini galleggiano
\usepackage{float}

\usepackage{color}

\usepackage{caption}
% per caption ad immagini in tab annidiate
\usepackage{subcaption}

\usepackage{hyperref} % lasciare per ultimo
%\hypersetup{colorlinks=true, linkcolor=blue, citecolor=black, plainpages=false, urlcolor=blue}
\hypersetup{colorlinks=true, linkcolor=black, citecolor=black, plainpages=false, urlcolor=blue}

% Usato nella copertina
\usepackage{wallpaper}



\usepackage{listings}
\lstset{
  showspaces=false,
  showstringspaces=false,
  basicstyle=\ttfamily,
  %numbers=left,
  %numbers=none,
  numberstyle=\small,
  mathescape
}

\usepackage{algpseudocode,algorithm,algorithmicx}
%\newcommand*\DNA{\textsc{dna}}
%
%\newcommand*\Let[2]{\State #1 $\gets$ #2}
%\algrenewcommand\algorithmicrequire{\textbf{Precondition:}}
%\algrenewcommand\algorithmicensure{\textbf{Postcondition:}}






% Simboli matematici
\usepackage{amsthm}
\usepackage{amssymb}
\usepackage{amsmath}
%\usepackage{amsfonts}
%\usepackage{mathabx} % per \topdoteq
%\usepackage{mathtools, amsthm}


% Comodo per evidenziare zone da modificare
\usepackage{todonotes} 

% Lorem ipsum...
\usepackage{lipsum} 


 
% ================================ %
%          Cose Personali          %
% ================================ %

% Alcune comodita' logico/matematiche
\let\ep\epsilon
\let\b\bullet
\let\iff\Leftrightarrow
\newcommand{\viff}{\Updownarrow}
\let\impl\Rightarrow

\newcommand{\abs}[1]{\lvert #1 \lvert}
\newcommand{\norm}[1]{\lvert\lvert #1 \lvert\lvert}
%% \newcommand{\Norm}[1]{\Big\lvert \Big\lvert #1 \Big\lvert \Big\lvert}


\newcommand\N{\ensuremath{\mathbb{N}}}
\newcommand\R{\ensuremath{\mathbb{R}}}
\newcommand\Z{\ensuremath{\mathbb{Z}}}
\renewcommand\O{\ensuremath{\emptyset}}
\newcommand\Q{\ensuremath{\mathbb{Q}}}
\newcommand\C{\ensuremath{\mathbb{C}}}

\newcommand\E{\ensuremath{\mathbb{E}}}
\newcommand\T{\ensuremath{\mathbb{T}}}
\renewcommand\inf{\ensuremath{\infty}}

\newtheorem{theorem}{Theorem}[section]




% ================================ %
%            Il Documento          %
% ================================ %
\begin{document}

% ================================ %
%        Creo Prima Pagina         %
% ================================ %

% TODO enable
%% \ThisCenterWallPaper{0.95}{polloPallido}
%% % Intestazione
%% % TODO Migliorare esteticamente
%% \begin{titlepage}
%%  	\centering
%%   \Huge{\textbf{TODO}}\\
%%  	[30mm]
%%  	\centering
%%   \Huge{\textbf{Student: Tristano Munini}}\\
%%  	%[25mm]
%%   %\raggedright
%%   %\Large{\textbf{Corso:}}\\
%%   %\Large{\textbf{TODO}}\\
%%   % TODO add "Relatore"
%%  	[125mm]
%%  	\centering
%%   \LARGE{\underline{\textbf{YEAR 2019-2020}}}\\
%% \end{titlepage}

% TODO enable
%% %% ================================ %
%% %%              Indice              %
%% %% ================================ %
%% \tableofcontents
%% \thispagestyle{empty}
%% \cleardoublepage
%% \setcounter{page}{1}


% ================================ %
%      Qua Inizia La Tesina        %
% ================================ %
%\abstract{
%  \lipsum[1]
%}

%\pagenumbering{gobble} % TODO REMOVE


%%%%\input{./chapters/x.tex}
%%%!TEX TS-program = pdflatex
%!TEX root = main.tex
%!TEX encoding = UTF-8 Unicode

\section{Introduction}
This work aims to be an initial approach to Implicit Surface Rendering with Sphere Tracing, a particular kind of ray tracing.
The reasons behind the topic choice are several:
in the last years real-time ray tracing has regained a lot of interest because can generate incredible images and the modern hardware can run it smoothly;
also these algorithm are being approximated by neural networks, and the topic of Neural Rendering is gaining more and more attention.
Since nowadays neural networks are being used inside or at the end of the graphic pipeline, e.g. DLSS, and ray tracing is being integrated to achieve realistic real-time reflections, it's plausible that in the future ray tracing will be approximated and/or optimized with deep learning techniques.

So with this work I wanted to understand the basics of implicit surface rendering and to implement a C++ version of the encountered algorithms.
This will give a good starting point to comprehend papers like \cite{nglod} in which is required an understanding of both the machine learning and ray tracing fields.
The developed codebase will be used in future for integrating machine learning algorithm and for learning CUDA.



%%\clearpage
%!TEX TS-program = pdflatex
%!TEX root = main.tex
%!TEX encoding = UTF-8 Unicode

\section{Signed Distance Functions}
\todo[inline]{TODO}

%%\clearpage
%%%!TEX TS-program = pdflatex
%!TEX root = main.tex
%!TEX encoding = UTF-8 Unicode

\section{Conclusions}
\todo[inline]{TODO}

%%\clearpage

%% ================================ %
%%           Bibliografia           %
%% ================================ %
\cleardoublepage
\printbibliography

\end{document}
