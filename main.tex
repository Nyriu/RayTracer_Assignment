\documentclass{article}
%documentclass[draft]{article}

%\usepackage[italian]{babel}
\usepackage[utf8]{inputenc}


\usepackage{graphicx} % Immagini fantastiche e...
\graphicspath{        % dove trovarle
  {./images/},
  {./images/2D/},
}

%% Bibliografia
\usepackage{csquotes}
\usepackage{biblatex}
\addbibresource{bibl.bib}


% Grafici direttamente in latex
\usepackage{tikz}
\usetikzlibrary{shapes,positioning,calc}
\colorlet{lightgray}{gray!20}

\usepackage{rotating} % per tabella ruotata
\usepackage{makecell}
%\usepackage[showframe=true]{geometry}
\usepackage{changepage}

% Immagini galleggiano
\usepackage{float}

\usepackage{color}

\usepackage{caption}
% per caption ad immagini in tab annidiate
\usepackage{subcaption}

\usepackage{hyperref} % lasciare per ultimo
%\hypersetup{colorlinks=true, linkcolor=blue, citecolor=black, plainpages=false, urlcolor=blue}
\hypersetup{colorlinks=true, linkcolor=black, citecolor=black, plainpages=false, urlcolor=blue}

% Usato nella copertina
\usepackage{wallpaper}



\usepackage{listings}
\lstset{
  showspaces=false,
  showstringspaces=false,
  basicstyle=\ttfamily,
  %numbers=left,
  %numbers=none,
  numberstyle=\small,
  mathescape
}

\usepackage{algpseudocode,algorithm,algorithmicx}
%\newcommand*\DNA{\textsc{dna}}
%
%\newcommand*\Let[2]{\State #1 $\gets$ #2}
%\algrenewcommand\algorithmicrequire{\textbf{Precondition:}}
%\algrenewcommand\algorithmicensure{\textbf{Postcondition:}}






% Simboli matematici
\usepackage{amsthm}
\usepackage{amssymb}
\usepackage{amsmath}
%\usepackage{amsfonts}
%\usepackage{mathabx} % per \topdoteq
%\usepackage{mathtools, amsthm}


% Comodo per evidenziare zone da modificare
\usepackage{todonotes} 

% Lorem ipsum...
\usepackage{lipsum} 


 
% ================================ %
%          Cose Personali          %
% ================================ %

% Alcune comodita' logico/matematiche
\let\ep\epsilon
\let\b\bullet
\let\iff\Leftrightarrow
\newcommand{\viff}{\Updownarrow}
\let\impl\Rightarrow

\newcommand{\abs}[1]{\lvert #1 \lvert}
\newcommand{\norm}[1]{\lvert\lvert #1 \lvert\lvert}
%% \newcommand{\Norm}[1]{\Big\lvert \Big\lvert #1 \Big\lvert \Big\lvert}


\newcommand\N{\ensuremath{\mathbb{N}}}
\newcommand\R{\ensuremath{\mathbb{R}}}
\newcommand\Z{\ensuremath{\mathbb{Z}}}
\renewcommand\O{\ensuremath{\emptyset}}
\newcommand\Q{\ensuremath{\mathbb{Q}}}
\newcommand\C{\ensuremath{\mathbb{C}}}

\newcommand\E{\ensuremath{\mathbb{E}}}
\newcommand\T{\ensuremath{\mathbb{T}}}
\renewcommand\inf{\ensuremath{\infty}}

\newtheorem{theorem}{Theorem}[section]




% ================================ %
%            Il Documento          %
% ================================ %
\begin{document}

% ================================ %
%        Creo Prima Pagina         %
% ================================ %

% TODO enable
%% \ThisCenterWallPaper{0.95}{polloPallido}
%% % Intestazione
%% % TODO Migliorare esteticamente
%% \begin{titlepage}
%%  	\centering
%%   \Huge{\textbf{TODO}}\\
%%  	[30mm]
%%  	\centering
%%   \Huge{\textbf{Student: Tristano Munini}}\\
%%  	%[25mm]
%%   %\raggedright
%%   %\Large{\textbf{Corso:}}\\
%%   %\Large{\textbf{TODO}}\\
%%   % TODO add "Relatore"
%%  	[125mm]
%%  	\centering
%%   \LARGE{\underline{\textbf{YEAR 2019-2020}}}\\
%% \end{titlepage}

% TODO enable
%% %% ================================ %
%% %%              Indice              %
%% %% ================================ %
%% \tableofcontents
%% \thispagestyle{empty}
%% \cleardoublepage
%% \setcounter{page}{1}


% ================================ %
%      Qua Inizia La Tesina        %
% ================================ %
%\abstract{
%  \lipsum[1]
%}

%\pagenumbering{gobble} % TODO REMOVE


%%%%\input{./chapters/x.tex}
%%%!TEX TS-program = pdflatex
%!TEX root = main.tex
%!TEX encoding = UTF-8 Unicode

\section{Introduction}
\todo[inline]{
Where I got the idea (and motivation) for this. \\
Objectives. \\
What I've achieved. \\
Why ray tracing is interesting (and where can be used)(?) \\
Here just one introductory page.
}

%%\clearpage
%!TEX TS-program = pdflatex
%!TEX root = main.tex
%!TEX encoding = UTF-8 Unicode

\section{Signed Distance Functions}
\todo[inline]{TODO}

%%\clearpage
%%%!TEX TS-program = pdflatex
%!TEX root = main.tex
%!TEX encoding = UTF-8 Unicode

\section{Conclusions}
\todo[inline]{TODO}

%%\clearpage

%% ================================ %
%%           Bibliografia           %
%% ================================ %
\cleardoublepage
\printbibliography

\end{document}
